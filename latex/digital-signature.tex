\documentclass[a4paper]{article}
\usepackage{graphicx}
\usepackage{twocolpceurws}

\title{The Digital Signature and the X.509/OpenPGP Authentication Models}

\author{Dashamir Hoxha \\ dashohoxha@gmail.com}

\institution{University "Ismail Qemali" of Vlora \\ Department of Computer Science}


\begin{document}
\maketitle

\begin{abstract}
This article explains what is a Digital Signature, why it is an
important part of the Digital Identity, and how it works. Then it
describes the authenticity and social problems related to the usage of
the Digital Signature. It explains as well the two authentication
models, X.509 and OpenPGP, that can be used to solve these
authenticity problems. Finally it makes a comparison between these two
authentication models and their features and tries to explain why the
OpenPGP model is better.
\end{abstract}

\section{Introduction}

\subsection{What is the Digital Signature}

Digital Signature means some digital data that are attached to a
digital document, which cannot be falsified, and which guaranty the
integrity and the authenticity of the document.

The "integrity" means that the document has not been changed/corrupted
since the time that it was digitally signed, either intentionally or by
mistake. The "authenticity" means that we can verify and be sure about
the person that signed the document (the author of the document).

So, the Digital Signature on digital documents has the same purpose as
the hand-written signature on the hard-copy (printed) documents.

The digital signature is not related to a scanned version of a
hard-copy document which is hand-signed, or to a faxed paper document
which is hand-signed, or to a scanned image of the hand-signature
which is included to a document or attached to an email. All these
methods seem to be intuitive for a digital signature, however they do
not fully guaranty the integrity and the authenticity of a document,
therefore cannot be used as valid digital signatures.

\subsection{Why the Digital Signature is Important}

Without being able to sign digital documents, they can never be
considered official, because we cannot be sure that they are original
and we cannot be sure who is the real author (they can be corrupted
and manipulated). So, despite using computers, digital systems and
digital documents, we always have to rely on the hard copies of the
documents and keep them around for official purposes, since we can't
fully trust the digital documents. This means that we will never be
able to build totally digital systems for institutions and
organizations, free from papers and hard-copy documents.

For example if a citizen has to interact with a governmental
institution and sends some documents online, nevertheless he still has
to submit the hard copies of the documents in person, since we can't
rely on the authorship and correctness of the digital documents.

Or if a person sends a document by email to a bank, the bank cannot
rely on it, since it cannot be fully sure about the authenticity (real
authorship) of the document, that it is not manipulated or corrupted
somehow, or that it is not just a trick or deception.

Only the Digital Signature can guaranty the identity of the author of
a document and establish a secure relationship between the people and
the digital documents. So, it is an essential tool for
enabling/supporting the digital identity, for establishing trust and
security on the digital world, and for building a digital society
(digital governance, digital business, etc.).


\section{How the Digital Signature Works}

The Digital Signature is based on the hash functions and the so called
asymmetric key cryptography (private/public key pairs).

\subsection{What is a Hash Function}

The job of a hash function is to digest (process) an electronic
document and to generate from it an extract. No matter how big is the
document, the extract has always the same fixed size. Two different
documents cannot produce the same extract. A document that is changed
even by a single character will produce a different extract after
being digested by the hash function.

\subsection{What is Asymmetric Key Cryptography}

Different from the symmetric cryptography, which uses the same key for
both encryption and decryption, asymmetric cryptography uses one key
for encryption and a different one for decryption.  Each person has a
pair of encryption keys; one of them is private (secret) and is known
only by the person himself, and the other key is public and is known
by everybody. A message that is encrypted by one of the keys, can be
decrypted only by the other key of the pair. It is almost impossible
to find the private key from the public key that is known by
everybody.

The algorithms that are used for generating a pair of private/public
keys and for encrypting and decrypting a message are based on the
arithmetic of large prime numbers and calculations with residue
classes. It is not difficult to understand them, however this is not
the proper place to explain such details. It is enough to know that
the asymmetric key cryptography is thought to be quite secure and
unbreakable.

\subsection{Signing a Digital Document}

To sign a digital document, these steps are followed:
\begin{enumerate}
  \item The digital document is digested by the hash function and a
    digital extract is produced.
  \item The digital extract of the document is encrypted with the
    private key of the author.
  \item This encrypted extract of the document is the digital
    signature and it can either be appended to the original document,
    or can be saved as a separate file.
\end{enumerate}

\subsection{Verifying the Digital Signature of a Document}

The verification of the digital signature of a document is done like
this:
\begin{enumerate}
  \item The digital document is digested by the hash function and its
    digital extract is produced.
  \item The digital signature of the document is decrypted with the
    public key of the author. This gives us the digital extract of the
    original document.
  \item The digital extract of the current document (from the first
    step) and the digital extract of the original document (from the
    second step) are compared. If they are the same, then the
    signature is good and the document is original. Otherwise the
    signature is bad and the authenticity of the document cannot be
    guarantied (most probably it has been corrupted, intentionally or
    by error).
\end{enumerate}

If the two digital extracts compared on the third step are equal, this
means that the original document is unmodified (since the digital
extract has not changed) and it verifies the authenticity of the
author (since his public key is able to decrypt the digital signature
correctly).

If these digital extracts are not equal, either the content of the
document has been corrupted (by error or intentionally), or the author
of the document is not the one who claims to be, or both. Any of these
reasons is enough to discard the document as invalid.

\subsection{A Concrete Example}

Email is a kind of digital document, and it can be signed digitally.
Actually it is the document that is most widely used with a digital
signature nowadays. This is probably due to the fact that the Internet
of today is not secure, and emails can be faked easily, and one cannot
be completely sure about its authenticity, unless it is digitally
signed.

Suppose that Alice sends an email to Bob. She signs this email using
her private key. Then Bob verifies the signature using the public key
of Alice. If the verification is successful, then Bob can be sure
that this email cannot have been signed except with the private key of
Alice.  Since only Alice has her private key, then only she can be the
signer (and hence the author) of the message.


\section{Authenticity Verification}

\subsection{Where to Find the Public Key}

Consider again the example of the last section, where Alice sends an
email to Bob. Where can Bob find the public key of Alice, so that he
can verify the authenticity of her message?

There are several ways that Bob can get her public key. Maybe Alice
gave it to him directly, using a removable media or sending it as an
attachment. Maybe Alice published it on her website and Bob got it
from there. Maybe Alice published it on some public key server and Bob
retrieved it from there (and this is the most common case in
practice).

A Public Key Server (PKS) is like a directory server (a dictionary),
where you can look up and retrieve the public key of a given person.
Alice can upload her public key on a PKS, and Bob (or anyone else that
needs to verify her signature) can look up and retrieve this key from
there.

Actually the public key of a person is stored in a digital document
that contains also the identity of a person (name, email, address,
organization, etc.). This document is called Digital Certificate (or
Identity Certificate, or Public Key Certificate). It is the Digital
Certificate that is uploaded to a PKS and retrieved from it, and it is
the Digital Certificate that makes the relation (connection) between
the digital identity of a person and his public key.

\subsection{The Problem of Authenticity}

Here we are faced with a problem. If Bob retrieved the digital
certificate of Alice from a PKS, how can he be sure that it is
authentic? How can he be sure that this certificate was uploaded there
by Alice and the public key in it really belongs to Alice?  Probably
somebody else uploaded that digital certificate there instead of
Alice, with the identity of Alice but with a different public key.

This is actually a social problem, not a technical one, and it can be
solved by social means. Bob can actually call Alice and make sure that
the ID of her key is the same as the one that he got from the PKS.  Or
probably Alice gave Bob a business card where she has also written the
ID of her public key, so Bob can check this ID with the ID of the key
that he retrieved from the PKS and make sure that it is correct.

However most of the time we communicate with people that we have never
met before and we have no idea who they are. It can be that "Alice" is
just a fake identity (a nickname or a fake name, not the name of a
real person). Or maybe somebody else uploaded the certificate instead
of Alice, pretending to be Alice, and the key in the certificate does
not really belong to Alice (is a fake public key).

If Bob has never met Alice, then how can he be sure about her real
identity? How can he be sure that Alice is a real person and that the
messages that he gets are really coming from her and not from somebody
pretending to be her? In other words, how can Bob be sure that the
digital certificate of Alice, that he gets from the PKS, is authentic?

Just verifying that the signature of the message is correct is not
enough.  We need to verify also that the digital certificate that was
used for the signature is authentic.

Again, this is a social problem and cannot be solved only by technical
means.  It can be solved only by a combination of social and technical
procedures.

\subsection{Verifying and Signing Digital Certificates}

Suppose that Chloe has checked the digital certificate of Alice and is
sure that:
\begin{enumerate}
  \item Alice is a real person and the digital identity on her digital
    certificate corresponds to her real-life identity and is correct.
  \item The public key in the digital certificate is the correct one
    (the one that belongs to Alice).
\end{enumerate}

How can Chloe check and verify the digital identity of Alice (first
point above)? If Chloe does not know Alice personally, she can ask to
meet her in person, check her identity documents (passport, identity
card, driver license, etc.)  and make sure that the digital identity
of Alice (name, birthday, etc.)  corresponds to the real-life
identity. For checking and verifying the public key (second point
above), Chloe has to get from Alice the fingerprint or ID of her
public key, compare it with the one on the digital certificate, and
make sure that it is the same.

Now that Chloe has verified that the digital certificate of Alice is
authentic, she can sign it. A digital certificate is just a digital
document, so it can be signed with a digital signature.

By signing the digital certificate of Alice, Chloe testifies that it
is correct and valid, which means that the digital identity is
authentic and the public key really belongs to Alice. The digital
signature of Chloe also guaranties that the information on the digital
certificate has not been changed since the time that she verified and
signed it.

\subsection{Introducers and Certification Authorities (CAs)}

If Bob has full trust on Chloe about checking and verifying the
information of digital certificates, then he can be sure that the
digital certificate of Alice is authentic and valid, without having to
check and verify it himself.

So, Bob asserts (derives) the validity/authenticity of the digital
certificate of Alice by trusting a third party, which is Chloe. Bob
can trust as well any other digital certificates that Chloe has
signed.  In such a case Chloe is called an "introducer" for Bob.

If Chloe verifies and signs a lot of digital certificates and a lot of
people trust the certificates signed by Chloe, then Chloe is called a
Certification Authority (CA).


\section{The Hierarchical (X.509) Authentication Model}

The X.509 authentication model is a hierarchical one. The digital
certificate of a person is verified and signed by a certification
authority (CA), the digital certificate of this CA is verified and
signed by a higher level CA, and so on until we reach a root CA, whose
digital certificate is self-signed (has signed himself his own digital
certificate).

For example, if Bob receives a message signed with the digital
certificate of Alice, he will notice that this digital certificate is
verified and signed by CA1, which in turn is verified and signed by
CA2, which is verified and signed by RCA (a root CA). Bob just has to
check that the certificate of the root CA is correct (valid and
authentic), and then he has to trust that each of RCA, CA1 and CA2
have done the verification and signing properly. He doesn't have to
check and verify the certificate of Alice directly. This chain
verification is usually done automatically by the software that Bob
uses.

The digital certificate (public key) of the root authority has to be
widely known and easily verifiable. And also Bob has to trust it
(actually it turns out that Bob does not have much choice on this,
because other people have decided that Bob should trust it). The
validation of a certificate is based on the trust that Bob has that
the root CA and each of the CAs have done their job properly (checking
and verifying the certificates of the next level).

CAs are usually commercial, but large institutions or government
entities may have their own CAs as well. There are about 50 root CAs
that are known worldwide.


\section{The Web-Of-Trust (OpenPGP) Authentication Model}

The OpenPGP standard uses a non-hierarchical, decentralized
authentication model that is called Web-Of-Trust. 

\subsection{Self-Signing Your Own Digital Certificate}

In the OpenPGP model each person acts as a root CA and first of all
self-signs his own digital certificate (to protect it from any
modification and forgery). For example Alice signs her own certificate
and Bob signs his own.

\subsection{Verifying and Signing Certificates of the Others}

Second, each of them can sign the certificates of the people, which
they have personally checked and verified. Verification includes both
making sure that the digital identity matches the real-life identity
of the person, and making sure that the public key in the certificate
is the correct one that belongs to this person. This certificate
verification and signing can be mutual as well, for example Alice
signs the certificate of Bob, and Bob signs the certificate of Alice.

When Alice signs the certificate of Bob, usually she makes public this
signature by uploading the signed certificate on a PKS. This lets
everybody know that she has checked and verified the digital
certificate of Bob and that she guaranties that this certificate is
authentic and valid.

\subsection{Deciding Whom To Trust}

Next, each person decides who are the people that he can trust about
making correct verification of others' certificates, and how much he
can trust them. The trust levels that are defined by the OpenPGP
standard are: unknown (default), none, marginal, full, ultimate.
These trust values are not about how trustable is this person in the
real life, but rather about the ability of the person to make correct
verification of digital certificates, before signing them.

For example the trust value 'marginal' means that you believe that
this person sometimes may not check and verify carefully the details
of a certificate, before signing it. The trust value 'full' means that
you believe that this person is very careful when signing
certificates. The trust value 'ultimate' means that you believe that
this person is so careful when checking and signing certificates,
that he almost never makes mistakes.

The trust level that one assigns to a person is subjective and can be
different from one person to another. For example Alice may have full
trust on Bob, however Chloe may think that Bob can be trusted only
marginally. The trust level is also private, which means that it is
relevant only to the person who assigns it, and it is not published on
any servers.

\subsection{Deciding About the Validity/Authenticity of a Certificate }

The figure shows a web of trust rooted at Alice. The graph illustrates
who has signed who's certificate.

\begin{figure}[ht]
\begin{center}
\includegraphics[width=9cm]{wot.jpg}
\caption{Web of Trust }
\label{wot}
\end{center}
\end{figure}

Alice is sure that the certificates of Blake and Dharma are valid,
since she has verified and signed them herself.

If Alice has full trust on Dharma, then she would consider valid the
certificates of Chloe and Fransis as well. She has not verified them
herself, but Dharma has verified and signed them and Alice has full
trust on the ability of Dharma to correctly verify and sign digital
certificates.

In case that Alice has only marginal trust on Blake and Dharma, then
she cannot be really sure about the validity of the Francis'
certificate, although Dharma has signed it. However, she can be almost
sure about the validity of the Chloe's certificate. Both Blake and
Dharma have verified and signed it, so the possibility of both of them
being deceived (or corrupted, mistaken) is small.

\subsection{Calculating the Validity/Authenticity of a Certificate}

The decision on which certificate can be considered fully valid, or
partially valid, or non-valid, is actually done automatically by the
software that is used for verifying the signature. The software makes
this decision based on who has signed who's certificate, on the trust
value assigned to each of the people on the web of trust, and
applying certain rules that are used to calculate the validity
(authenticity) of a certificate. Such a rule can be for example: a
certificate that is signed by at least three marginally trusted people
can be considered fully valid.

The validation rules are customizable and can be different for each
person, in order to fit the security requirements of everybody. For
example, if Alice does not have any high security needs, and she lives
in a friendly (not hostile) environment, then she may decide that even
two marginally trusted signatures are enough to consider a certificate
fully valid. However, if she has high security requirements and she
lives in a rather hostile environment, then she can decide that at
least five marginally trusted signatures should be required, so that a
certificate that she has not verified herself can be considered valid.
In this case, since she has decided to depend less on the
verifications done by the others, she has to do more verifications
on her own.

\subsection{Digital Notaries}

Sometimes there are people who do a great many of verification and
signing of the others' digital certificates, even on a full time
bases, and they are trusted by everybody (or at least a lot of
people). These people play the role of a CA (Certification Authority)
in the OpenPGP model.

Such people can be for example the head of the IT department in a
company or institution. Or they can be people approved, verified and
authorized by the government to offer this kind of service to the
citizens. In this case they can also be called Digital Notaries
and they may offer other Digital Services as well, besides verifying
and signing digital certificates.

The Digital Notaries can also be held responsible in front of law for
the correctness and truthfulness of the verifications and signatures
that they make (as well as for other digital or non-digital services
that they may offer). This accountability can be very useful for
increasing their responsibility, as well as for increasing the trust
of people on them and the health and reliability of the web-of-trust
system as a whole.

\section{Comparing the X.509 and OpenPGP Authentication Models}

The digital certificates of both standards, X.509 and OpenPGP, are
very similar in content and they are based on the same principles (of
asymmetric cryptography, private/public key pair, etc.). However their
authentication models are different and not interoperable. This means
that a digital certificate that is recognized as valid and authentic
by one of them, can not be recognized as such by the other.

However both of them can be used concurrently (at the same time)
without interfering with each-other. This means that a person can have
one certificate of type X.509 and another of type OpenPGP at the same
time, and use either one of them or the other, as needed.  This is
also facilitated by the fact that most of the software that are used
for digital signatures support both of these standards.

\subsection{Inflexible vs Flexible and Versatile}

If we compare the structure of the authentication models of the X.509
and the OpenPGP standards, we will notice that the first one closely
resembles a tree (is hierarchical, like the structure of the
private/governmental organizations), while the second one resembles a
web or mesh (like the structure of the Internet).

A mesh is a much more flexible structure than a tree, because a tree
structure is just a special case of a mesh structure.

In the web-of-trust authentication model of OpenPGP there can be CAs
as well (like in the case of Digital Notaries that we have discussed
previously).  If many people choose to fully trust the same CA for
checking the validity/authenticity of the others certificates (and
they all configure their own copies of the OpenPGP client software to
trust that CA), then the OpenPGP model acts just like the X.509
model. In fact, the web model of OpenPGP is a proper superset of the
hierarchical model of X.509. There is no situation in the X.509 model
that cannot be handled exactly the same way in the OpenPGP model. But
OpenPGP can do much more.

In the X.509 model the set of trusted root CAs is fixed and
predetermined. The users have no choice and can make no decision
whether to trust them or not. This is so true that these CAs are
"baked into" the major software that uses digital certificates
(e.g. browsers). On the OpenPGP model, on the other hand, the users
can decide themselves whom to trust and how much to trust them.

\subsection{Centralized vs Decentralized and Distributed}

We can notice as well that the hierarchical model is centralized,
while the web-of-trust model is distributed and decentralized. This is
related to who is responsible for ensuring the correctness,
authenticity and validity of the certificates, the security,
trustability and reliability of the whole system, etc.

On the hierarchical model this responsibility falls on some central
authority (the root CA), and on the sub-authorities (CAs) that it
approves. On the web-of-trust model this responsibility falls on
everybody participating on the system, since each of them helps to
verify and validate the certificates of the others. So, on the
web-of-trust model, each person that holds a digital certificate is
verified by the others and helps to verify the others at the same
time. This is a more democratic model, that encourages the
responsibility and the participation of the citizens.

\subsection{Vulnerable vs Robust and Reliable}

The decentralized/distributed model is also more robust and reliable
than the hierarchical model.

The hierarchical model has just a single point-of-failure that has to
be watched, protected and guarded very carefully, since it is a clear
target of attack. This is the root CA. In case that its security is
compromised or corrupted some day, then the security of the whole
system is compromised and all of the digital certificates of the
system are rendered invalid.

This doesn't have to be a technical failure (for example some hackers
breaking into the system), it can be a social corruption as well (and
this can be even more likely than a technical failure). This risk is
amplified by the fact that most of the CAs are commercial. Matt Blaze
once made the cogent observation that commercial CAs will protect you
against anyone who that CA refuses to accept money from!

The distributed model, on the other hand, is much more difficult to
corrupt because each participant is a little CA on its own. Maybe some
of them can be corrupted for some time, but it is quite difficult to
corrupt many or most of them at the same time. In any case, there can
be inflicted only local damages, the whole system will survive the
attack, and with time it can auto-correct and heal itself gradually.


\section{Summary}

It is quite easy to understand the concept of Digital Signatures and
the basics of how it works. The Digital Signature is so important that
it will become an inevitable part of our future digital societies.

A very important aspect of the digital signature is verification of
its authenticity. It happens that this is more a social problem than a
technical one, so it can be solved correctly only by the right
combination of social and technical means.

Currently, there are two models (or infrastructures) for solving the
authentication problem. One of them is the Hierarchical model (X.509
standard), and the other one is the Web-Of-Trust model (OpenPGP
standard). The Web-Of-Trust model is more flexible and advanced than
the Hierarchical model, but it requires that everybody that
participates in it takes responsibility and makes decisions for
himself.

However I think that the Web-Of-Trust is the right approach, because
the personal privacy and security are, by definition, personal
responsibilities, and they cannot be outsourced.


\section{References}

\begin{itemize}
  \item http://en.wikipedia.org/wiki/Digital\_signature
  \item http://en.wikipedia.org/wiki/Public\_key
  \item http://en.wikipedia.org/wiki/Digital\_certificate
  \item http://en.wikipedia.org/wiki/X.509
  \item http://en.wikipedia.org/wiki/Web\_of\_trust
  \item http://www.youdzone.com/signature.html
  \item http://www.gnupg.org/gph/en/manual.html
  \item http://www.cryptnet.net/fdp/crypto/keysigning\_party/en/keysigning\_party.html
  \item http://www.openpgp.org/technical/whybetter.shtml
  \item http://enigmail.mozdev.org/
  \item http://www.gpg4win.org/
\end{itemize}

\end{document}
